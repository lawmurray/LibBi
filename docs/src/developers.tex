\section{Introduction}

LibBi consists of two major components\index{components}:
\begin{itemize}
\item A \emph{library} that provides classes and functions for simulation,
  filtering, smoothing, optimising and sampling state-space models, as well as
  auxiliary functionality such as memory management, I/O, numerical computing
  etc. It is written in C++ using a generic programming paradigm.
\item A \emph{code generator} that parses the LibBi modelling language,
  constructs and optimises an internal model representation, and generates C++
  code for compilation against the library. It also includes the command-line
  interface through which the user interacts with LibBi. It is written in Perl
  using an object-oriented paradigm.
\end{itemize}

Related projects might also be considered components. These include
\emph{OctBi} for GNU Octave and \emph{RBi} for R. Both allow the querying,
collation and visualisation of results output by LibBi.

Developing LibBi involves adding or modifying functionality in one or more of
these components\index{components}. A typical exercise is adding block and
action types to the code generator to introduce new features, such as support
for additional probability density functions\index{probability\,density}. This
may also involve implementing new functionality in the library that these
blocks and actions can use. Another common task is adding a new inference
method to the library, then adding or modifying the code generator to produce
the client code to call it.

\section{Setting up a development environment}

When developing LibBi, it is recommended that you do not install it in the
usual manner as a system-wide Perl module. The recommended approach is to work
from a local clone of the LibBi Git repository. This way modifications take
effect immediately without additional installation steps, and changes are
easily committed back to the repository.

\subsection{Obtaining the source code}

LibBi is hosted on GitHub at
\href{https://github.com/libbi/LibBi}{https://github.com/libbi/LibBi}. It is
recommended that you sign up for an account on GitHub and fork the repository
there into your own account. You can then clone your forked repository to your
local machine with:
\begin{cmdcode}
git clone https://github.com/username/LibBi.git
\end{cmdcode}
You should immediately add the original LibBi repository upstream:
\begin{cmdcode}
cd LibBi
git remote add upstream https://github.com/libbi/LibBi.git
\end{cmdcode}
after which the following command will pull in changes committed to the main
repository by other developers:
\begin{cmdcode}
git fetch upstream
\end{cmdcode}
To contribute your work back to the main LibBi repository, you can use the
\emph{Pull Request} feature of GitHub.

Similar instructions apply to working with OctBi and RBi, which are held in
separate repositories under
\href{https://github.com/libbi}{https://github.com/libbi}.

When running the command \bitt{libbi}, you will want to make sure that you are
calling the version in the Git repository that you have just created. The
easiest way to do this is to add the \bitt{LibBi/script} directory to your
\bitt{\$PATH} variable. LibBi will find the other files itself (it uses the
base directory of the \bitt{libbi} script to determine the location of other
files).

\subsection{Using Eclipse\label{Using_Eclipse}}

LibBi is set up to use the Eclipse IDE
(\href{http://www.eclipse.org}{http://www.eclipse.org}). From a vanilla
install of the Eclipse IDE for C/C++, a number of additional plugins are
supported, and recommended. These are, along with the URLs of their update
sites:
\begin{itemize}
\item Eclox (Doxygen integration) \href{http://download.gna.org/eclox/update}{http://download.gna.org/eclox/update},
\item EPIC (Perl integration) \href{http://e-p-i-c.sf.net/updates/testing}{http://e-p-i-c.sf.net/updates/testing}, and
\item Perl Template Toolkit Editor \href{http://perleclipse.com/TTEditor/UpdateSite}{http://perleclipse.com/TTEditor/UpdateSite}.
\end{itemize}
All of these plugins may be installed via the \textsf{Help > Install New Software...}
menu item in Eclipse, in each case entering the update site URL given above.

The EGit plugin is bundled with all later versions of Eclipse and can be used
to obtain LibBi from GitHub. Use the \textsf{File > Import...} menu item, then
\textsf{Git > Projects from Git}, \textsf{URI}, and enter the URL of your
forked repository on GitHub. Follow the prompts from there.

\index{style\,guide} A custom code style is available
in the \bitt{eclipse/custom\_code\_style.xml} file, and custom code templates
in \bitt{eclipse/custom\_code\_templates.xml}. These can be installed via the
\textsf{Window > Preferences} dialog, using the \textsf{Import...} buttons
under \textsf{C/C++ > Code Style > Formatter} and \textsf{C/C++ > Code Style >
  Code Templates}, respectively. With these installed, \textsf{Ctrl + Shift +
  F} in an editor window automatically applies the prescribed style.

%%%%
\section{Building documentation}

This user manual can be built, from the base directory of LibBi, using the
command:
\begin{cmdcode}
perl MakeDocs.PL
\end{cmdcode}
You will need \LaTeX\ installed. The manual will be built in PDF format at
\bitt{docs/pdf/index.pdf}. Documentation of all actions, blocks and clients is
extracted from their respective Perl modules as part of this process.

The Perl components of LibBi are documented using POD (``Plain Old
Documentation''), as is conventional for Perl modules. The easiest way to
inspect this is on the command line, using, e.g.
\begin{cmdcode}
perldoc lib/Bi/Model.pm
\end{cmdcode}
A PDF or HTML build of the Perl module documentation is not yet available.

The C++ components of LibBi are documented using Doxygen. HTML documentation
can be built by running \bitt{doxygen} in the base directory of LibBi, then
opening the \bitt{docs/dev/html/index.html} file in a browser.

%%%%
\section{Building releases}

LibBi is packaged as a Perl module. The steps for producing a new release are:
\begin{enumerate}
\item Update \bitt{VERSION.md}.
\item Update \bitt{MANIFEST} by removing the current file, running \bitt{perl
  MakeManifest.PL} and inspecting the results. Extraneous files that should
  not be distributed in a release are eliminated via regular expressions in
  the \bitt{MANIFEST.SKIP} file.
\item Run \bitt{perl Makefile.PL} followed by \bitt{make dist}.
\end{enumerate}
On GitHub, the last step is eliminated in favour of tagging the repository
with version numbers and having archives automatically built by the service.

%%%%
\section{Developing the code generator}

The code generator component is implemented in Perl using an object-oriented
paradigm, with extensive use of the Perl Template Toolkit (TT) for producing
C++ source and other files.

\subsection{Actions and blocks}

A user of LibBi is exposed only to those blocks and actions which they
explicitly write in their model specification file. Developers must be aware
that beyond these \textit{explicit} blocks, additional \textit{implicit}
blocks are inserted according to the actions specified by the user. In
particular, each action has a preferred parent block type. If the user, in
writing their model specification, does not explicitly place an action within
its preferred block type, the action will be implicitly wrapped in it anyway.

For example, take the model specification:
\begin{bicode}
sub transition \{
  x ~ uniform(x - 1.0, x + 1.0)
  y ~ gaussian(y, 1.0)
\}
\end{bicode}
The \actionref{uniform} and \actionref{gaussian} actions prefer different
parent block types. These additional block types will be inserted, to give the
equivalent of the user having specified:
\begin{bicode}
sub transition \{
  sub uniform_ \{
    x ~ uniform(x - 1.0, x + 1.0)
  \}
  sub gaussian_ \{
    y ~ gaussian(y, 1.0)
  \}
\}
\end{bicode}

For the developer, then, actions and blocks are always written in tandem.
Note the underscore suffix on \bitt{uniform\_} and \bitt{gaussian\_} is a
convention that marks these as blocks for internal use only -- the user is not
intended to use them explicitly (see the \secref{Style_guide}{style guide}).

It is worth familiarising yourself with this behaviour, and other
transformations made to a model, by using the \clientref{rewrite} command.

To add a block:
\begin{enumerate}
\item Choose a name for the block.
\item Create a Perl module \bitt{Bi::Block::\textit{name}} in the
  file \bitt{lib/Bi/Block/\textit{name}.pm}. Note that Perl module
  ``CamelCase'' naming conventions should be ignored in favour of LibBi block
  naming conventions here.
\item Create a new TT template
  \bitt{share/tt/cpp/block/\textit{name}.hpp.tt}. When rendered, the template
  will be passed a single variable named \bitt{block}, which is an object of
  the class created in the previous step.
\end{enumerate}

As implementation details of the Perl module and template are subject to
change, it is highly recommended that you copy and modify an existing, similar
block type, as the basis for your new block type.

To add an action:
\begin{enumerate}
\item Choose a name for the action.
\item Create a Perl module \bitt{Bi::Action::\textit{name}} in the
  file \bitt{lib/Bi/Action/\textit{name}.pm}. Note that Perl module
  ``CamelCase'' naming conventions should be ignored in favour of LibBi action
  naming conventions here.
\item If necessary, create a new TT template
  \bitt{share/tt/cpp/action/\textit{name}.hpp.tt}. During rendering, the
  template will be passed a single variable named \bitt{action}, which is an
  object of the class created in the previous step. This template is not
  required; the template of the containing block may generate all the
  necessary code.
\end{enumerate}

\subsection{Clients}

To add a client:
\begin{enumerate}
\item Consider whether an existing client can be modified or a new client
  should be added. For an existing client, edit the corresponding Perl module
  \bitt{lib/Bi/Client/\textit{name}.pm}. For a new client, choose a name
  and create a new Perl module \bitt{lib/Bi/Client/\textit{name}.pm}. Again,
  LibBi client naming conventions should be favoured over Perl module naming
  conventions.
\item Consider whether an existing template can be modified or a new template
  should be created. For an existing template, edit the corresponding template
  in \bitt{share/tt/cpp/client/\textit{template}.cpp.tt}. For a new template:
\begin{enumerate}
\item Create files \bitt{share/tt/cpp/client/\textit{template}.cpp.tt} and
  \bitt{share/tt/cpp/client/\textit{template}.cu.tt}. Note that the second
  typically only needs to \bitt{\#include} the former, the \bitt{*.cu}
  file extension is merely needed for a CUDA-enabled compile.
\item Modify the Perl module to select this template for use where
  appropriate.
\item Add the template name to the list given at the top of
  \bitt{share/tt/build/Makefile.am.tt} to ensure that a build fule is generated
  for it.
\end{enumerate}
\end{enumerate}

\subsection{Designing an extension\label{Designing_an_extension}}

Complex C++ code should form part of the library, rather than being included
in TT templates. The C++ code generated by templates should typically be
limited to arranging for a few function calls into the library, where most of
the implementation is done.

When writing templates for client programs, consider that it is advantageous
for the user that they can change command-line options without the C++ code
changing, so as to avoid a recompile. This can be achieved by writing runtime
checks on command-line options in C++ rather than code generation-time checks
in TT templates. This need not be taken to the extreme, however: clear and
simple template code is preferred over convoluted C++!

\subsection{Documenting an extension\label{Documenting_an_extension}}

Reference documentation for actions, blocks and clients is written in the Perl
module created for them, in standard Perl POD. Likewise, parameters for
actions and blocks, and command-line options for clients, are enumerated in
these. The idea of this is to keep implementation and documentation together.

POD documentation in these modules is automatically converted to \LaTeX\ and
incorporated in the \secref{User_Reference}{User Reference} section of this
manual.

\subsection{Developing the language\label{Developing_the_language}}\index{language}

The LibBi modelling language is specified in the files \bitt{share/bi.lex}
(for the lexer) and \bitt{share/bi.yp} (for the parser). The latter file makes
extensive callbacks to the \bitt{Bi::Parser} module. After either of the two
files are modified, the automatically-generated \bitt{Parse::Bi} module must
be rebuilt by running \bitt{perl MakeParser.PL}.

\subsection{Style guide\label{Developer_style_guide}}\index{style\,guide}

Further to the \secref{Style_guide}{style guide} for users, the
following additional recommendations pertain to developers:
\begin{itemize}
\item Action, block, client and argument names are all lowercase, with
  multiple words separated by '\_' (the underscore). Uppercase may be used in
  exceptional cases where this convention becomes contrived. A good example is
  matrix arguments, which might naturally be named with uppercase letters.
\item Actions, blocks and arguments that are not meant to be used explicitly
  should be suffixed with a single underscore.
\end{itemize}

\section{Developing the library}

The library component is implemented in C++ using a generic programming
paradigm. While classes and objects are used extensively, it is not
object-oriented \textsl{per se}. In particular:
\begin{itemize}
\item Class inheritance is used for convenience of implementation, not
  necessarily to represent ``is a'' relationships.
\item Polymorphism is seldom, if ever, employed.
\item Static dispatch is favoured strongly over dynamic dispatch for
  reasons related to performance. Virtual functions are not used.
\end{itemize}

\subsection{Header files}

Header files are given two extensions in LibBi:
\begin{enumerate}
\item \bitt{*.hpp} headers may be safely included in any C++ (e.g.
\bitt{*.cpp}) or CUDA (\bitt{*.cu}) source files,
\item \bitt{*.cuh} headers may only be safely included in CUDA (\bitt{*.cu})
source files. They include CUDA C extensions.
\end{enumerate}

Note that not all \bitt{*.hpp} headers can be safely included in \bitt{*.cu}
files either, due to CUDA compiler limitations, particularly those headers
that further include Boost headers which make extensive use of
templates. Efforts have been made to quarantine such incidences from the CUDA
compiler, and support is improving, but mileage may vary.

\subsection{Pseudorandom reproducibility\label{Pseudorandom_reproducibility}}\index{pseudorandom\,number\,generation}

It is important to maintain reproducibility of results under the same random
number seed\index{random\,number\,seed}, typically passed using the
\bitt{--seed} option on the command line. Real time impacts should be
considered, such as favouring static scheduling\index{static\,scheduling} over
dynamic scheduling\index{dynamic\,scheduling} for OpenMP\index{OpenMP} thread
blocks. Consider the following:
\begin{cppcode}
Random rng;

#pragma omp parallel for
for (int i = 0; i < N; ++i) \{
  x = rng.uniform();
  \(\ldots\)
\}
\end{cppcode}

\bitt{Random} maintains a separate pseudorandom number
generator\index{pseudorandom\,number\,generation} for each OpenMP thread. If
dynamically scheduled, the above loop gives no guarantees as to the number of
variates drawn from each generator, so that reproducibility of results for a
given seed is not guaranteed. Static scheduling should be enforced in this
case to ensure reproducibility:
\begin{cppcode}
Random rng;

#pragma omp parallel for schedule(static)
for (int i = 0; i < N; ++i) \{
  x = rng.uniform();
  \(\ldots\)
\}
\end{cppcode}

\begin{tip}
To avoid this particular case, static scheduling is enabled by default. You
should not override this with dynamic scheduling in an OpenMP loop.
\end{tip}

A more subtle consideration is the conditional generation of
variates. Consider the following code, evaluating a
Metropolis-Hastings\index{Metropolis-Hastings} acceptance criterion:
\begin{cppcode}
alpha = (l1*p1*q2)/(l2*p2*q1);
if (alpha >= 1.0 || rng.uniform() < alpha) \{
  accept();
\} else \{
  reject();
\}
\end{cppcode}
Here, \bitt{rng.uniform()} is called only when \bitt{alpha >= 1.0} (given
operator short-circuiting). We might prefer, however, that across multiple
runs with the same seed, the pseudorandom number generator is always in the
same state for the $n$th iteration, regardless of the acceptance criteria
across the preceding iterations. Moving the variate generation outside the
conditional will fix this:
\begin{cppcode}
alpha = (l1*p1*q2)/(l2*p2*q1);
u = rng.uniform();
if (alpha >= 1.0 || u < alpha) \{
  accept();
\} else \{
  reject();
\}
\end{cppcode}

\subsection{Shallow copy, deep assignment}

Typically, classes in LibBi follow a ``shallow copy, deep assignment''
idiom. The idiom is most apparent in the case of vector and matrix types. A
copy of a vector or matrix object is shallow -- the new object merely contains
a pointer to the memory buffer underlying the existing object. An assignment
of a vector or matrix is deep, however -- the contents of the memory buffer
underlying the existing object is copied into that of the new object.

Often the default copy constructor is sufficient to achieve a shallow copy,
while an override of the default assignment operator may be necessary to
achieve deep assignment.

\emph{Generic} copy constructors and assignment operators are also
common. These are templated overloads of the default versions which accept, as
arguments, objects of some class other than the containing class. When a
generic copy constructor or assignment operator is used, the default copy
constructor or assignment operator should always be overridden also.

\subsection{Coding conventions}

The following names are used for template parameters (where \bitt{n} is an
integer):
\begin{itemize}
\item \bitt{B} for the model type,
\item \bitt{Tn} for scalar types,
\item \bitt{Vn} for vector types,
\item \bitt{Mn} for matrix types,
\item \bitt{Qn} for pdf types,
\item \bitt{S} for type lists,
\item \bitt{L} for location (host or device),
\item \bitt{CL} for location of a cache,
\item \bitt{PX} for parents type,
\item \bitt{CX} for coordinates type,
\item \bitt{OX} for output type.
\end{itemize}

\subsection{Style guide}

In brief, the style of C++ code added to the library should conform to what is
already there. If using Eclipse, the custom code style file (see
\secref{Using_Eclipse}{Using Eclipse}) is a good start.

C++ code written in the library should conform to the ``Kernighan and Ritchie''
style, indenting\index{indenting} with two spaces and never tabs.
