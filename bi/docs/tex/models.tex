\subsection{{\sf model}\label{model}}

Declare a model.

\subsubsection*{Synopsis\label{model_Synopsis}}
\begin{alltt}{\sf
    model \textsl{Name} \{
      \(\ldots\)
    \}
}\end{alltt}


\subsubsection*{Description\label{model_Description}}

A \textsf{model} statement is the outermost statement of a model specification,
declaring and naming the model.

Immediately within the \textsf{model} statement, the following named top-level blocks should
be used to specify the model:
\begin{itemize}
\item \textsf{\hyperref[hyper][parameter]{parameter}}, specifying the prior density over parameters,
\item \textsf{\hyperref[hyper][initial]{initial}}, specifying the prior density over initial conditions,
\item \textsf{\hyperref[hyper][transition]{transition}}, specifying the transition density, and
\item \textsf{\hyperref[hyper][observation]{observation}}, specifying the observation density.
\end{itemize}

The following named top-level blocks are optional, but may be required in order to use particular methods:
\begin{itemize}
\item \textsf{\hyperref[hyper][proposal_parameter]{proposal\textunderscore{}parameter}}, specifying the proposal density over parameters,
\item \textsf{\hyperref[hyper][proposal_initial]{proposal\textunderscore{}initial}}, specifying the proposal density over initial conditions,
\item \textsf{\hyperref[hyper][lookahead_transition]{lookahead\textunderscore{}transition}}, specifying a lookahead density to accompany the transition density, and
\item \textsf{\hyperref[hyper][lookahead_observation]{lookahead\textunderscore{}observation}}, specifying a lookahead density to accompany the observation density.
\end{itemize}

\subsection{{\sf dim}\label{dim}\index{dim}}

Declare a dimension.

\subsubsection*{Synopsis\label{dim_Synopsis}}
\begin{alltt}{\sf
    dim \textsl{name}(100, 'cyclic')
    dim \textsl{name}(size = 100, boundary = 'cyclic')
}\end{alltt}


\subsubsection*{Description\label{dim_Description}}

A \textsf{dim} statement declares a dimension with a given size and boundary
condition.

A \textsf{dim} statement may only be used at the top level of the model
specification. Dimensions must be declared before any variables are declared
along them.

\subsubsection*{Arguments\label{dim_Arguments}}

\begin{itemize}
\item \textsf{size} (position 0, mandatory)

Length of the dimension.

\item \textsf{boundary} (position 1, default 'none')

Boundary condition of the dimension. Valid values are:

\begin{itemize}

\item \textsf{'none'}

No boundary condition.

\item \textsf{'cyclic'}

Cyclic boundary condition; all indices are taken modulo the \textsf{size} of
the dimension.

\end{itemize}
\end{itemize}

\subsection{{\sf input}\index{input}, {\sf noise}\index{noise}, {\sf obs}\index{obs}, {\sf param}\index{param} and {\sf state}\index{state}\label{var}}

Declare an input, noise, observed, parameter or state variable.

\subsubsection*{Synopsis\label{var_Synopsis}}

\begin{alltt}{\sf
    state x
    state x[i]
    state x(io = 0)
    state x[i](io = 0)
}\end{alltt}

\subsubsection*{Description\label{var_Description}}

Any of these statements declare a variable of the given type, along the list
of dimensions given between the square brackets.

These statements may only be used at the top level of the model
specification. Dimensions along which a variable extends must be declared
prior to the declaration of the variable.

\subsubsection*{Arguments\label{var_Arguments}}

\begin{itemize}
\item \textsf{io} (default 1)

Should the variable be included in input and output buffers? Excluding an
uninteresting variable from outputs by setting this to 0 will save disk space
and I/O time.

\item \textsf{tmp} (default 0)

Is the variable merely a temporary? This is used internally to flag
temporary variables that are created to hold intermediate results, but which
need not be represented in, for example, means and covariances of the joint
distribution over all variables. It may have occasional use externally.

\end{itemize}

\subsection{{\sf const}\label{const}\index{const}}

Declare a constant.

\subsubsection*{Synopsis\label{const_Synopsis}}
\begin{alltt}{\sf
    const \textsl{name} = \textsl{constant_expression}
}\end{alltt}

\subsubsection*{Description\label{const_description}}

A {\sf const} statement declares a constant, the value of which is evaluated
using the given constant expression\index{constant expression}. The constant
may then be used, by name, in other expressions.

A constant may be declared anywhere in a model specification, but always has
global scope.

\subsection{{\sf inline}\label{inline}\index{inline}}

Declare an inline expression.

\subsubsection*{Synopsis\label{inline_synopsis}}
\begin{alltt}{\sf
    inline \textsl{name} = \textsl{expression}
}\end{alltt}

\subsubsection*{Description\label{inline_description}}

An {\sf inline} statement declares an inline expression\index{inline
  expression}. The inline may then be used, by name, in other expressions, as
long as it will not violate any constraints on those expressions (e.g. an
inline expression named within a constant expression must itself be a constant
expression).

An inline expression may be declared anywhere in a model specification, but
always has global scope.

\subsection{{\sf sub}\label{sub}\index{sub}}

\subsection{{\sf do} and {\sf then}\label{do_then}\index{do.. then}}
